\documentclass[a4paper, 12pt]{report}
\usepackage[portuguese]{babel}
\usepackage[utf8]{inputenc}
\usepackage[T1]{fontenc}
\usepackage{graphicx}
\usepackage{listings}
\usepackage{xcolor}
\usepackage{hyperref}
\usepackage{tocloft}
\usepackage{amsmath}
\usepackage{geometry}
\usepackage{fancyhdr}
\usepackage{booktabs}
\usepackage{caption}

% Configurações de página
\geometry{margin=2.5cm}
\pagestyle{fancy}
\fancyhf{}
\rhead{\thepage}
\lhead{Projeto de Gerenciamento de Antenas}
\rfoot{}

% Configurações de código
\definecolor{codegreen}{rgb}{0,0.6,0}
\definecolor{codegray}{rgb}{0.5,0.5,0.5}
\definecolor{codepurple}{rgb}{0.58,0,0.82}
\definecolor{backcolour}{rgb}{0.95,0.95,0.92}

\lstdefinestyle{codeStyle}{
    backgroundcolor=\color{backcolour},   
    commentstyle=\color{codegreen},
    keywordstyle=\color{magenta},
    numberstyle=\tiny\color{codegray},
    stringstyle=\color{codepurple},
    basicstyle=\ttfamily\footnotesize,
    breakatwhitespace=false,         
    breaklines=true,                 
    captionpos=b,                    
    keepspaces=true,                 
    numbers=left,                    
    numbersep=5pt,                  
    showspaces=false,                
    showstringspaces=false,
    showtabs=false,                  
    tabsize=2,
    frame=single
}

\lstset{style=codeStyle}

% Configurações do índice
\renewcommand{\cfttoctitlefont}{\hfill\Large\bfseries}
\renewcommand{\cftaftertoctitle}{\hfill}
\renewcommand{\cftsecleader}{\cftdotfill{\cftdotsep}}
\setlength{\cftbeforesecskip}{4pt}

% Metadados
\title{
    \vspace{2cm}
    \textbf{Projeto de Gerenciamento de Antenas}\\
    \large Sistema de Controle e Detecção de Interferências
}
\author{
    Alexandre Barbosa\\
}
\date{\today}

\begin{document}

\maketitle
\tableofcontents
\lstlistoflistings

\chapter{Introdução}
Este documento descreve um sistema completo de gerenciamento de antenas implementado em C, com todas as funcionalidades de inserção, remoção, visualização e detecção de interferências.

\chapter{Implementação}
\section{Estruturas de Dados}
\begin{lstlisting}[language=C, caption=Definição da estrutura Antena]
typedef struct Antena {
    char frequencia;  // Frequência (A-Z)
    int x, y;         // Coordenadas
    struct Antena *prox; // Próxima antena
} Antena;
\end{lstlisting}

\section{Funções Principais}

\subsection{Criação de Antena}
\begin{lstlisting}[language=C, caption=Função criarAntena]
Antena* criarAntena(char frequencia, int x, int y) {
    Antena* nova = (Antena*)malloc(sizeof(Antena));
    nova->frequencia = frequencia;
    nova->x = x;
    nova->y = y;
    nova->prox = NULL;
    return nova;
}
\end{lstlisting}

\subsection{Inserção}
\begin{lstlisting}[language=C, caption=Função inserirAntena]
void inserirAntena(Antena **lista, char frequencia, int x, int y) {
    Antena *nova = criarAntena(frequencia, x, y);
    nova->prox = *lista;
    *lista = nova;
    printf("Antena inserida com sucesso!\n");
}
\end{lstlisting}

\subsection{Remoção}
\begin{lstlisting}[language=C, caption=Função removerAntena]
void removerAntena(Antena **lista, int x, int y) {
    Antena *atual = *lista, *anterior = NULL;
    while (atual != NULL && (atual->x != x || atual->y != y)) {
        anterior = atual;
        atual = atual->prox;
    }
    if (atual == NULL) {
        printf("Antena nao encontrada!\n");
        return;
    }
    if (anterior == NULL) *lista = atual->prox;
    else anterior->prox = atual->prox;
    free(atual);
    printf("Antena removida com sucesso!\n");
}
\end{lstlisting}

\subsection{Visualização do Mapa}
\begin{lstlisting}[language=C, caption=Função exibirMapa]
void exibirMapa(Antena *lista) {
    char mapa[MAP_SIZE][MAP_SIZE];
    for (int i = 0; i < MAP_SIZE; i++) {
        for (int j = 0; j < MAP_SIZE; j++) {
            mapa[i][j] = '.';
        }
    }
    
    Antena *atual = lista;
    while (atual != NULL) {
        if (atual->x >= 0 && atual->x < MAP_SIZE && 
            atual->y >= 0 && atual->y < MAP_SIZE) {
            mapa[atual->y][atual->x] = atual->frequencia;
        }
        atual = atual->prox;
    }
    
    printf("\nMapa de Antenas:\n");
    for (int i = 0; i < MAP_SIZE; i++) {
        for (int j = 0; j < MAP_SIZE; j++) {
            printf("%c ", mapa[i][j]);
        }
        printf("\n");
    }
}
\end{lstlisting}

\subsection{Detecção de Interferências}
\begin{lstlisting}[language=C, caption=Função detectarInterferencia]
void detectarInterferencia(Antena *lista) {
    Antena *a1, *a2;
    int interferencia = 0;
    printf("\nAntenas com interferencia:\n");
    for (a1 = lista; a1 != NULL; a1 = a1->prox) {
        for (a2 = a1->prox; a2 != NULL; a2 = a2->prox) {
            if (a1->frequencia == a2->frequencia) {
                int distancia = abs(a1->x - a2->x) + abs(a1->y - a2->y);
                if (distancia <= 2) {
                    printf("Antenas em (%d, %d) e (%d, %d) com frequencia '%c'\n", 
                           a1->x, a1->y, a2->x, a2->y, a1->frequencia);
                    interferencia = 1;
                }
            }
        }
    }
    if (!interferencia) printf("Nenhuma interferencia detectada.\n");
}
\end{lstlisting}

\section{Função Principal}
\begin{lstlisting}[language=C, caption=Função main]
int main() {
    Antena *lista = NULL;
    int opcao, x, y;
    char freq;
    
    do {
        printf("\nMenu:\n");
        printf("1 - Inserir Antena\n");
        printf("2 - Remover Antena\n");
        printf("3 - Listar Antenas\n");
        printf("4 - Exibir Mapa\n");
        printf("5 - Detectar Interferencias\n");
        printf("0 - Sair\n");
        printf("Escolha: ");
        scanf("%d", &opcao);
        
        switch (opcao) {
            case 1:
                printf("Frequencia (A-Z): ");
                scanf(" %c", &freq);
                printf("Coordenadas (x y): ");
                scanf("%d %d", &x, &y);
                inserirAntena(&lista, freq, x, y);
                break;
            case 2:
                printf("Coordenadas para remover (x y): ");
                scanf("%d %d", &x, &y);
                removerAntena(&lista, x, y);
                break;
            case 3:
                imprimirAntenas(lista);
                break;
            case 4:
                exibirMapa(lista);
                break;
            case 5:
                detectarInterferencia(lista);
                break;
            case 0:
                printf("Saindo...\n");
                break;
            default:
                printf("Opcao invalida!\n");
        }
    } while (opcao != 0);
    
    return 0;
}
\end{lstlisting}

\chapter{Conclusão}
O sistema implementa todas as funcionalidades requeridas de forma eficiente, utilizando estruturas de dados adequadas e algoritmos otimizados para a detecção de interferências.

\end{document}